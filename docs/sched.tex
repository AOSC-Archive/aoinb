\documentclass[UTF8,zihao=-4]{ctexart}
\usepackage[a4paper]{geometry}
\usepackage{fullpage}
\usepackage{titling}
\usepackage{diagbox}
\usepackage{xcolor}
\usepackage{amsmath}
\usepackage{isomath}
\usepackage{cases}
% \usepackage{mathptmx}
\setmainfont{Linux Libertine O}
\setCJKmainfont{Source Han Serif CN}
\title{打包调度机制}
\date{}
\author{gumblex}
\setlength{\droptitle}{-2cm}
\ctexset{
  section/format = \Large\bfseries,
  subsection/format = \large\bfseries
}
\renewcommand\thesubsection{}

\begin{document}
\maketitle

\section{定义}
分布式打包调度首先需要解析依赖,对依赖做拓扑排序,然后把同级的依赖包分配到不同机器去完成编译。其中,依赖解析和拓扑排序有较为通用的算法。我们在此解决的问题主要是如何去将包分配给各台不同的机器。

设包 $i$ 的工作量为 $w_i$。设某特定软件包(如 \textit{glibc})的工作量为 1。

对于机器 $j$,设其编译速度(编译单位工作量所需时间)为 $v_j$、CPU 个数为 $c_j$,且有内存限制 $M_j$、硬盘限制 $D_j$。

用机器 $j$ 编译包 $i$ 时,可以观察到实际使用时间(real)$t_{ij}$、CPU使用时间(user+sys)$T_{ij}$,以及使用 $c_j$ 个 CPU 时最大内存使用量 $m_i(c_j)$ 和最大硬盘使用量 $d_i$。相同架构的机器编译所使用的内存和硬盘基本相同。

\section{参数估计}

\subsection{工作量和性能}

用机器 $j$ 编译包 $i$ 时,有
$$ \frac{w_i}{v_j} = T_{ij} \enspace \Rightarrow \enspace
\log(w_i) - \log(v_j) = \log(T_{ij})$$

列出所有机器编译每个包所需CPU时间的数值,以 $\log(w_i)$ 和 $\log(v_j)$ 作为未知数,建立稀疏矩阵,可使用最小二乘法求解该线性系统,得到每个 $w_i$ 和 $v_j$ 的估计值。

\subsection{并行效率}

定义包 $i$ 在 $c$ 核机器上编译的并行效率为 $$ P_i(c) = \max \{0, \frac{T_i/t_i-1}{c-1} \} $$

实际编译时间即为
\begin{equation}
\label{eq_ti} t_{ij} = \frac{w_i}{v_j (1 + (c-1) \cdot P_i(c))}
\end{equation}

假设 $ P_i(c) $ 为一次函数 $ k_i c + b_i $,用最小二乘法线性回归可求出参数 $k_i$ 和 $b_i$。


若存在 $x$ 使 $P_i(x) = 0$,则认为函数即为 $P_i(c) = 0$,表示编译过程为单线程。

假设 $ m_i(c) $ 为一次函数 $ k_i c + b_i $,用最小二乘法线性回归可求出参数 $k_i$ 和 $b_i$。

\section{工作分配}

使用上述方法可以估算出每个包的工作量 $w_i$(默认为 1)、包的并行效率 $P_i(c)$(默认参数为所有包的平均)、每台机器的效率 $v_j$(需要进行基准测试)。用式 \eqref{eq_ti} 可估算出包 $i$ 在机器 $j$ 上编译的时间 $t_{ij}$。

设目标函数 $z$ 为所需最大时长。决策变量 $a_{ij}$ 为是否将包 $i$ 分配给机器 $j$。总共有 $X$ 个包、$Y$ 台机器。

列出整数线性规划问题:$\min z$
\begin{subnumcases}{\label{eq_lp} \text{s.t.}}\displaystyle
\sum^Y_{j=1} a_{ij} = 1, & i = 1, 2, ..., X \\
z \ge \sum^X_{i=1} a_{ij} t_{ij}, & j = 1, 2, ..., Y \\
a_{ij} m_i(c_j) \le M_j \\
a_{ij} d_i \le D_j \\
a_{ij} \in \{0, 1\}
\end{subnumcases}

其中,约束条件 (\ref{eq_lp}a) 表示同一个包只能分配给一台机器;(\ref{eq_lp}b) 表示目标函数最大时长要大于每台机器上所有任务的时长之和;(\ref{eq_lp}c)、(\ref{eq_lp}d) 表示不把任务分配给不满足内存和硬盘要求的机器;(\ref{eq_lp}e) 表示决策变量 $a_{ij}$ 为0-1变量。

\end{document}
